\documentclass[a4paper,12pt]{jarticle}%articleをjarticleにすると,figure1が図1になる
%\documentclass[dvipdfmx]{jlreq}

\usepackage{amsmath,amssymb}
\usepackage[dvipdfmx]{graphicx}
\usepackage{geometry}
\geometry{top=20mm,bottom=20mm,left=15mm,right=15mm}
\usepackage{float}% "H" = ここに図を強制的に配置,を使えるようにする
\usepackage{caption}%図や表の外の環境でcaptionを使うためのパッケージ(minipageで混ぜる時用)
% \usepackage{hyperref}%\refでクリックできるリンクがつくらしい(?)
% \usepackage{url} %URLを入れる用

\usepackage[hyphens]{url}
\usepackage{ruby}


\title{Aチーム\ruby{操作}{そうさ}\ruby{説明書}{せつめいしょ}}
\date{}
% 最初のページだけ消したい場合


\begin{document}
\maketitle
\thispagestyle{empty} %ページ番号を消す


\vspace{-0.5cm}
\begin{figure}[H]
    \centering
    \includegraphics[width=0.6\linewidth]{UI_2.png}
    \label{fig:1}
\end{figure}

\vspace{1cm}
\noindent
\begin{minipage}{0.48\textwidth}
    \begin{minipage}{0.15\textwidth}
        \includegraphics[width=\linewidth]{UI_2_左上_1.png}
    \end{minipage}
    \begin{minipage}{0.8\textwidth}
        \large アームAが\ruby{上下}{じょうげ}に\ruby{動}{うご}く。
    \end{minipage}
\end{minipage}
\hfill
\begin{minipage}{0.48\textwidth}
    \begin{minipage}{0.15\textwidth}
        \includegraphics[width=\linewidth]{UI_2_右上.png}
    \end{minipage}
    \begin{minipage}{0.8\textwidth}
        \large アームBが\ruby{上下}{じょうげ}に\ruby{動}{うご}く。
    \end{minipage}
\end{minipage}


\vspace{0.5cm}
\noindent
\begin{minipage}{0.48\textwidth}
    \begin{minipage}{0.15\textwidth}
        \includegraphics[width=\linewidth]{UI_2_左上_2.png}
    \end{minipage}
    \begin{minipage}{0.8\textwidth}
        \large アームAが\ruby{開閉}{かいへい}する。
    \end{minipage}
\end{minipage}
\hfill
\begin{minipage}{0.48\textwidth}
    \begin{minipage}{0.15\textwidth}
        \includegraphics[width=\linewidth]{UI_2_左上_2.png}
    \end{minipage}
    \begin{minipage}{0.8\textwidth}
        \large アームBが\ruby{開閉}{かいへい}する。
    \end{minipage}
\end{minipage}





\vspace{2cm}
\hspace{0.7cm}
\begin{minipage}{0.4\textwidth}
    \includegraphics[width=\linewidth]{UI_2_真ん中.png}
\end{minipage}
\begin{minipage}{0.8\textwidth}
    \large\hspace{0.8cm}ロボットが\ruby{前後}{ぜんご}\ruby{左右}{さゆう}に\ruby{動}{うご}く。
\end{minipage}






\vspace{2cm}
\noindent
\begin{minipage}{0.48\textwidth}
    \begin{minipage}{0.15\textwidth}
        \includegraphics[width=\linewidth]{UI_2_左下.png}
    \end{minipage}
    \begin{minipage}{0.8\textwidth}
        \large ロボットが90\textdegree\ruby{左回転}{ひだりかいてん}する。
    \end{minipage}
\end{minipage}
\hfill
\begin{minipage}{0.48\textwidth}
    \begin{minipage}{0.15\textwidth}
        \includegraphics[width=\linewidth]{UI_2_右下.png}
    \end{minipage}
    \begin{minipage}{0.8\textwidth}
        \large ロボットが90\textdegree\ruby{右回転}{みぎかいてん}する。
    \end{minipage}
\end{minipage}




\end{document}